\appendix
\section{Additional Related Work}
{\bf Spatial regression.} Spatial regression techniques can be used to explain data \cite{dunn1986applied,cleveland1988locally}. The idea behind regression techniques is to express an attribute that we are interested in as a dependent variable. This dependent variable can then be expressed in the form of parametric equation involving other attributes in the dataset as independent variables. The user of this system decides a dependent variable and a set of explanation variables. The resulting equation has coefficients assigned to each explanatory variable as well as a bias term. This results in a curve fitting problem. The curve fitting problem is solved using regression i.e. the coefficient terms and the bias are iteratively adjusted until the sum of squared error between the predicted curve and the ground truth results in a minimum value. Regression techniques are widely used for spatial data analysis. However, they depend on predefined spatial partitioning and look at the data as a whole rather than from the perspective of a user defined observation.

{\bf Spatial autocorrelation.} Besides work related to explanation, there has been a lot of research in the area of spatial correlation. Much of the work in this area extends from multiple old works by Getis \cite{getis1991spatial,ord1995local,getis1996local,getis2002comparative,getis2007reflections}. The Getis Ord statistic \cite{ord1995local} for example is useful in showing us areas with high local spatial associations. The Moran's I statistic is useful for measuring the spatial heterogeneity of the data \cite{assuncao1999new,zhang2008use}. Moran's I is useful in hot spot analysis which can be viewed as a step in the way of finding explanations.


\section{Big Spatial Data Systems.}
{\bf Distributed data systems} Map Reduce \cite{dean2008mapreduce} is a framework for data processing which is designed for taking distributed and parallel computation into perspective. There are three main operations in a mapreduce process: map, shuffle and reduce. The map operation assigns a key to each element and performs any necessary transformations. The shuffle operation relocates the elements such that elements with the same key are nearby(since they are going to need each other in calculations). The reduce step performs a calculation on each element with the same key and returns the output. Spark \cite{shanahan2015large,zaharia2016apache} is a distributed and parallel processing framework. Spark uses a directed acyclic graph to perform calculations. Since the DAG created by Spark can have a lot of common nodes between tasks, the computational complexity of the operation is reduced compared to MapReduce.
Geospark~ \cite{yu2015geospark} is a framework for performing several spatial operations on data in Apache Spark. It also has a component which helps in data visualizations \cite{yu2018src}.